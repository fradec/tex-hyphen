% title: Hyphenation patterns for Occitan
% copyright: Copyright (C) 2016 Claudio Beccari
% notice: This file is part of the hyph-utf8 package.
%     See http://www.hyphenation.org/tex for more information.
% language:
%     name: Occitan
%     tag: oc
% version: 0.1 2016/02/04
% authors:
%   -
%     name: Claudio Beccari
%     contact: claudio.beccari (at) gmail.com
% licence:
%     - This file is available under any of the following licences:
%     -
%         name: MIT
%         url: https://opensource.org/licenses/MIT
%         text: >
%             Permission is hereby granted, free of charge, to any person
%             obtaining a copy of this software and associated documentation
%             files (the “Software”), to deal in the Software without
%             restriction, including without limitation the rights to use,
%             copy, modify, merge, publish, distribute, sublicense, and/or sell
%             copies of the Software, and to permit persons to whom the
%             Software is furnished to do so, subject to the following
%             conditions:
%
%             The above copyright notice and this permission notice shall be
%             included in all copies or substantial portions of the Software.
%
%             THE SOFTWARE IS PROVIDED “AS IS”, WITHOUT WARRANTY OF ANY KIND,
%             EXPRESS OR IMPLIED, INCLUDING BUT NOT LIMITED TO THE WARRANTIES
%             OF MERCHANTABILITY, FITNESS FOR A PARTICULAR PURPOSE AND
%             NONINFRINGEMENT. IN NO EVENT SHALL THE AUTHORS OR COPYRIGHT
%             HOLDERS BE LIABLE FOR ANY CLAIM, DAMAGES OR OTHER LIABILITY,
%             WHETHER IN AN ACTION OF CONTRACT, TORT OR OTHERWISE, ARISING
%             FROM, OUT OF OR IN CONNECTION WITH THE SOFTWARE OR THE USE OR
%             OTHER DEALINGS IN THE SOFTWARE.
%     -
%         name: LPL
%         version: 1
%         or_later: true
%         url: https://latex-project.org/lppl/
% ==========================================
% Patterns for the Occitan language; they are supposed to be valid
% for all the Occitan variants spoken and written in the wide area
% called “Occitanie” by the French. It ranges from the Val d'Aran
% within Catalunya, to the South Western Italian Alps encompassing 
% the southern half of the French pentagon.
%
% For more information please read the babel-occitan documentation.
%
\patterns{%
.anti1
.anti3m2n
.circu2m1
.e2x1
.para1i
.para1u
.proto1a
.proto1e
.proto1i
.proto1u
.su2b3lu
.su2b3r
3p2sic
3p2neu
1ï
1ü
a1ia
a1ie
a1io
a1iu
a1or.
i1or.
a1oira.
i1oira.
e1iu
1ii.
io1i
o1ia
o1ie
o1io
o1iu
1b
2bb
2bd
b2l
2bm
2bn
b2r
2bt
2bs
2b.
1c
2cc
c2h2
c2l
2cm
2cn
2cq
c2r
2cs
2ct
2cz
2c.
1ç
2ç.
2ch.
1d
2dd
2dg
2dm
d2r
2ds
2dv
2d.
1f
2ff
f2l
2fn
f2r
2ft
2f.
1g
2gg
2gd
2gf
g2l
2gm
2gn
g2r
2gs
g2ü2
2gv
2g.
1h
2hp
2ht
2h.
1j
1k
2kk
k2h2
1l
2lb
2lc
2ld
2lf
l3f2t
2lg
l2h
2lk
2ll
2lm
2ln
2lp
2lq
2lr
2ls
2lt
2lv
2l.
2lh.
1m
2mm
2mb
2mp
2ml
2mn
2mq
2mr
2mv
2m.
1n
2nb
2nc
2nç
2nd
2nf
n2h
2ng
2nj
2nl
2nm
2nn
2np
2nq
2nr
2ns
n2s3m
n2s3f
2nt
2nv
2nx
2n.
2nh.
2ns.
1p
p2h
p2l
2pn
2pp
p2r
2ps
2pt
2pz
2php
2pht
2p.
1q
1qu2
q2ü2
1r
2rb
2rc
2rç
2rd
2rf
2rg
r2h
2rl
2rm
2rn
2rp
2rq
2rr
2rs
2rt
r2ü
2rv
2rz
2r.
2rt.
.s2
1s
2s3ph
2ss
2s2tb
2s2tc
2s2td
2s2tf
2s2tg
2s2t3l
2s2tm
2s2tn
2s2tp
2s2tq
2s2ts
2s2tt
2s2tv
2s.
2st.
2sc.
2sb
2sc
2sd
2sf
2sg
s2h
2sj
2sk
2sl
2sm
2sn
2sp
2sr
2st
2sv
2sz
2sh.
1t
2tb
2tc
2td
2tf
t2g
t2h
t2j
t2l
t2r
2tm
2tn
2tp
2tq
2tt
2tv
t2z
2tg.
2tz.
2t.
1v
v2l
v2r
2vv
1x
2xt
2xx
2x.
1z
2z.
2'2
2b'
2ch'.
2ch''.
2c'
2d'
2f'
2g'
2h'
2j'
2k'
2l'.
2l''
2m'
2n'
2p'
2q'
2r'
2sh'
4s'.
4s''
2t'.
2t''
2v'.
2v''
2w'
2x'
2z'.
2z''
}
