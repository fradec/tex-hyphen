% title: Hyphenation patterns for Malayalam
% copyright: Copyright (C) 2016 Santhosh Thottingal
% notice: This file is part of the hyph-utf8 package.
%     See http://www.hyphenation.org/tex for more information.
% source: https://github.com/santhoshtr/hyphenation/
% language:
%     name: Malayalam
%     tag: ml
% version: 0.9.0 2016-01-16
% authors:
%   -
%     name: Santhosh Thottingal
%     contact: santhosh.thottingal (at) gmail.com
% licence:
%     - This file is available under any of the following licences:
%     -
%         name: MIT
%         url: https://opensource.org/licenses/MIT
%         text: >
%             Permission is hereby granted, free of charge, to any person
%             obtaining a copy of this software and associated documentation
%             files (the "Software"), to deal in the Software without
%             restriction, including without limitation the rights to use,
%             copy, modify, merge, publish, distribute, sublicense, and/or sell
%             copies of the Software, and to permit persons to whom the
%             Software is furnished to do so, subject to the following
%             conditions:
%
%             The above copyright notice and this permission notice shall be
%             included in all copies or substantial portions of the Software.
%
%             THE SOFTWARE IS PROVIDED "AS IS", WITHOUT WARRANTY OF ANY KIND,
%             EXPRESS OR IMPLIED, INCLUDING BUT NOT LIMITED TO THE WARRANTIES
%             OF MERCHANTABILITY, FITNESS FOR A PARTICULAR PURPOSE AND
%             NONINFRINGEMENT. IN NO EVENT SHALL THE AUTHORS OR COPYRIGHT
%             HOLDERS BE LIABLE FOR ANY CLAIM, DAMAGES OR OTHER LIABILITY,
%             WHETHER IN AN ACTION OF CONTRACT, TORT OR OTHERWISE, ARISING
%             FROM, OUT OF OR IN CONNECTION WITH THE SOFTWARE OR THE USE OR
%             OTHER DEALINGS IN THE SOFTWARE.
%     -
%         name: LGPL
%         version: 3
%         or_later: true
%         url: http://www.gnu.org/licenses/lgpl.html
%     -
%         name: GPL
%         version: 3
%         or_later: true
%         url: http://www.gnu.org/licenses/gpl.html
% hyphenmins:
%     generation:
%         left: 1
%         right: 1
% texlive:
%     babelname: malayalam
%     message: Malayalam hyphenation patterns
%     package: indic
%
\patterns{
% GENERAL RULE
% Do not break either side of ZERO-WIDTH JOINER  (U+200D)
2‍2
% Break on both sides of ZERO-WIDTH NON JOINER  (U+200C)
1‌1
% Break before or after any independent vowel.
1അ1
1ആ1
1ഇ1
1ഈ1
1ഉ1
1ഊ1
1ഋ1
1ൠ1
1ഌ1
1ൡ1
1എ1
1ഏ1
1ഐ1
1ഒ1
1ഓ1
1ഔ1
% Break after any dependent vowel, but not before.
ാ1
ി1
ീ1
ു1
ൂ1
ൃ1
െ1
േ1
ൈ1
ൊ1
ോ1
ൌ1
ൗ1
% Break before or after any consonant.
1ക
1ഖ
1ഗ
1ഘ
1ങ
1ച
1ഛ
1ജ
1ഝ
1ഞ
1ട
1ഠ
1ഡ
1ഢ
1ണ
1ത
1ഥ
1ദ
1ധ
1ന
1പ
1ഫ
1ബ
1ഭ
1മ
1യ
1ര
1റ
1ല
1ള
1ഴ
1വ
1ശ
1ഷ
1സ
1ഹ
% Do not break before anusvara, visarga
2ഃ1
2ം1
% Do not break either side of virama (may be within conjunct).
2്2
% Do not break left side of chillu
ന്2
ര്2
ള്2
ല്2
ക്2
ണ്2
2ന്‍
2ല്‍
2ള്‍
2ണ്‍
2ര്‍
2ക്‍
2ൺ
2ൻ
2ർ
2ൽ
2ൾ
2ൿ
}
